\documentclass[arial, greek, nologo, notitle, totpages]{europecv2013}
\usepackage{graphicx}
\usepackage[a4paper, top=1.2cm, left=1.2cm, right=1.2cm, bottom=2.5cm]{geometry}
\usepackage[greek, english]{babel}
\usepackage{url}
\usepackage[T1]{fontenc}


\ecvname{Παπαδόπουλος, Γεώργιος} % Sets your name to name.
\ecvaddress{Ι.Δροσοπούλου 152, Αθήνα, Ελλάδα - \foreignlanguage{english}{GR}-11256} % Sets the address to addr.
\ecvtelephone[697-9999999]{211-9999999} % Sets the telephone number to tel and, optionally, the mobile number to mobile.
\ecvemail[ηλεκτρονικό ταχυδρομείο : \foreignlanguage{english}{papadopoulos@company.gr}]{papadopoulos@company.gr}
\ecvhomepage{\href{http://www.papadopoulos.eu}{\foreignlanguage{english}{www.papadopoulos.eu}}}
\ecvlinkedin{\href{https://www.linkedin.com/in/papadopoulos}{\foreignlanguage{english}{LinkedIn search as George papadopoulos}}}

\ecvgender{Άρρεν} % Sets the gender to gender.
\ecvdateofbirth{30/05/1975} % Sets the date of birth to birth.
\ecvnationality{Ελληνική} % Sets the nationality to nationality.
\ecvfootername{Γεώργιος Παπαδόπουλος} % Sets your name as it appears in the footer. By default, the name in the footer is the same as the one specified with \ecvname. Use \ecvfootername if you want it to be different (for example, if you want to exchange the order of first name and last name).
\ecvfootnote{Για περισσότερες πληροφορίες επισκεφθείτε την ιστοσελίδα \foreignlanguage{english}{\textcopyright European Union, 2002-2015 | \url{http://europass.cedefop.europa.eu}}}


\begin{document}
\selectlanguage{greek}

\begin{europecv}

\ecvpersonalinfo[10pt]

\ecvposition{\large\textbf{Σκηνοθέτης}}

\ecvsection{Εργασιακή Εμπειρία}
%[Add separate entries for each experience. Start from the most recent.]

\ecvworkexperience{2014 - Σήμερα}{Ιδρυτής εταιρίας παραγωγής βίντεο και υπηρεσίες διακικτύου - \foreignlanguage{english}{papadopoulos.gr}}{\foreignlanguage{english}{papadopoulos.gr}}{Υπηρεσίες παραγωγής βίντεο και διαδικτύου}{Σκηνοθέτης, Παραγωγός}

\ecvworkexperience{2014 - Σήμερα}{Ανάθεση έργου για την \foreignlanguage{english}{backstage} κάλυψη της ψυχαγωγικής εκπομπής \foreignlanguage{english}{Papadopoulos Times}}{\foreignlanguage{english}{Papadopoulos Films}}{Τηλεοπτικές παραγωγές}{Σκηνοθέτης, Παραγωγός}

\ecvsection{Συμμετοχές}

\ecveducation{2003 - 2006}{Ηθοποιός}{Ανώτατη Σχολή Δραματικής Τέχνης}{}{}

\ecvsection{Εκπαίδευση και κατάρτιση}
%[Add separate entries for each course. Start from the most recent.]

\ecveducation{1996}{Απολυτήριο Ενιαίου Λυκείου}{22 Ενιαίο Λύκειο Αθηνών}{Μέση}{}

\ecvsection{Ατομικές δεξιότητες και ικανότητες}


\ecvmothertongue[20pt]{Ελληνικά}
\ecvlanguageheader{Λοιπές γλώσσες}
\ecvlanguage{Αγγλικά}{\ecvCOne}{\ecvCTwo}{\ecvCOne}{\ecvBTwo}{\ecvBTwo}
\ecvlanguage{Γαλλικά}{\ecvBOne}{\ecvBOne}{\ecvBOne}{\ecvBOne}{\ecvBOne}
\ecvlanguagefooter[10pt]{Επίπεδα}

\ecvitem{\large Άδεια οδήγησης}{Κατηγορία Β}

\ecvsection{Πρόσθετες πληροφορίες}

\ecvitem[10pt]{}{Οι στρατιωτικές μου υποχρεώσεις έχουν εκπληρωθεί.}

\ecvsection{Δείγματα δουλειάς}
% τα παρακάτω επαναλαμβάνονται για κάθε μία καταχώρηση, ξεκινώντας από την πιο πρόσφατη
\ecvitem{Σύνδεσμος}{Προσωπική ιστοσελίδα \foreignlanguage{english}{\url{http://www.papadopoulos.eu}}}

\ecvitem{Σύνδεσμος}{\foreignlanguage{english}{Trailer \url{https://www.youtube.com/watch?v=xxxxx}}}

\end{europecv}
\end{document}